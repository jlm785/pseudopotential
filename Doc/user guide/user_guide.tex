% Manual for the ATOM program
%
% To generate the printed version:
%
% pdflatex manual
%
%
\documentclass[11pt]{article}

\tolerance 10000
\textheight 24cm
\textwidth 16cm
\oddsidemargin 1mm
\topmargin -20mm

\parindent=0cm
\baselineskip=12pt
\parskip 5pt

\begin{document}

% TITLE PAGE
% --------------------------------------------------------------

\begin{titlepage}

\begin{center}
~
\vfill
\vspace{1cm}
{\Huge {\bf ATOM User Manual}}
\par\vspace{3cm}
\hrulefill
\par\vspace{3cm}
{\Large {\bf Version 6.0.5, October 2021}}
\par\vspace{2cm}
\hrulefill

{\Large Jos\'e Lu\'{\i}s Martins \\
Departamento de F\'{\i}sica, Instituto Superior T\'ecnico, Lisboa, Portugal

INESC MN, Lisboa, Portugal

\texttt{jlmartins@inesc-mn.pt}

\texttt{jose.l.martins@tecnico.ulisboa.pt}
}
vfill
\end{center}

\end{titlepage}
% END TITLE PAGE
% --------------------------------------------------------------

\tableofcontents
\newpage

\section{Introduction}
\label{sec:intro}

This is a modernized version of a old code that generates and tests pseudopotentials to be
used in electronic structure theory calculations.

The original code was written by Sverre Froyen at the University of California Berkeley. 
Phys. Rev. B 26, 1738 (1982), https://doi.org/10.1103/PhysRevB.26.1738

That code had major enhancements by Norman J. Troullier at the University of Minnesota, Minneapolis.
Phys. Rev. B 43, 1993 (1991), https://doi.org/10.1103/PhysRevB.43.1993

Since then the code has been maintained and improved by Jos\'e Lu\'{\i}s Martins.  Peter Schuster and Manuel Maria Alemany
wrote new plotting packages.  GGA were introduced by Carlos Balbas and José Soler in a collaboration
with the SIESTA code.

The present code is a "translation" from old FORTRAN (Fortran 70) to modern Fortran (90 -> present)
with detailed documentation by Jos\'e Lu\'{\i}s Martins, plus a few enhancements.

\section{Installation}
\label{sec:install}

\subsection{Step 1: Downloading and extracting the archive}
\label{sec:step1}

The code is available from Jos\'e Lu\'{\i}s Martins home page:

\noindent\texttt{https://fenix.tecnico.ulisboa.pt/homepage/ist13146/pseudopotential}

and from GitHub

\noindent\texttt{https://github.com/jlm785/}

If you downloaded the \texttt{atom-6.x.y.tgz} file (where \texttt{x} and \texttt{y}
are the minor version numbers) just extract it,

\noindent\texttt{\$ tar xzf atom-6.x.y.tgz}

and you will have a \texttt{atom-6.x.y} directory

If you cloned the git you alredy have the relevant code in the main directory.

\subsection{Step 2: Generating the documentation}
\label{sec:step2}

The documentation is in the \texttt{atom-6.x.y/Doc} directory.  It includes a short description of the key input file
\texttt{atom.dat} taken from the SIESTA documentation by Alberto Garcia, this file, and the means to generate a detailed
description of the code for developers.

For that detailed description you need doxygen (\texttt{https://www.doxygen.nl}) and 
graphviz (\texttt{https://graphviz.org/}), both are
available on most distributions.  To check that your computer has doxygen and graphviz installed
run the commands

\noindent\texttt{atom-6.x.y/Doc\$ dpkg -s doxygen}

\noindent\texttt{atom-6.x.y/Doc\$ dpkg -s graphviz}

on a Debian based distribution (ubuntu et al), or use equivalent tools (yum, rpm, dnf zypper,...) or the relevant GUI.

If they are not available just install them

\noindent\texttt{atom-6.x.y/Doc\$ sudo apt install doxygen}

\noindent\texttt{atom-6.x.y/Doc\$ sudo apt install graphviz}

again for debian based distros, or use equivalent tools (yum, rpm, dnf zypper,...) or the relevant GUI.

Finally run doxyfile in the doxy directory

\noindent\texttt{atom-6.x.y/Doc/doxy\$ doxygen Doxyfile}

If you know what you are doing you can edit \texttt{atom-6.x.y/Doc/doxy/Doxyfile}.

Finally it is useful to create a link to the file \texttt{atom-6.x.y/Doc/html/index.html}

\noindent\texttt{atom-6.x.y/Doc\$ ln -s html/index.html documentation.html}

Opening that link in your browser will allow you to see the documentation for every file in the present code.

\subsection{Step 3: Compiling the code}
\label{sec:step3}

The code has been tested with several Fortran compilers, \texttt{ifort} (from Intel oneAPI), 
\texttt{gfortran} (from gnu), 
\texttt{pgfortran} (from Portland), and even LLVM (experimental Intel compiler).

As the performance is not an issue for these ``simple'' one-dimensional (radial Schrödinger) calculations
just use the compiler that is more familiar for you.

The choice of compiler is in the \texttt{atom-6.x.y/Src/make.inc} file.  You may edit it for your convenience,
but in principle you have to make just two choices.  The first is to identify which CPU you are using,
as it narrows the choice of possible compilers.  This is done by commenting/uncommenting
the lines below \texttt{\#   compiler for the job}.

The second is the compiler you want, and is the key decision.  Agaim comment/uncommnet the lines below
\texttt{\#   Suggestions for compilers}.

For any other modification of the file it is assumed you know what you are doing...

To compile just go to \texttt{atom-6.x.y/Src} and type make

\noindent\texttt{atom-6.x.y\$ cd Src}

\noindent\texttt{atom-6.x.y/Src\$ make}

In a few seconds you should have the main library \texttt{libatom{\_}}\textit{compiler}\texttt{.a} 
and a few executables \texttt{atom{\_}}\textit{compiler}\texttt{.exe}, etc... 
where \textit{compiler} is the name of the chosen compiler. 
As you probably are not interested in comparing compilers, modify
the script \texttt{copyit} to simplify the names and relocate the executables.  In my case
I just copy them to \texttt{atom-6.x.y} removing the compiler name.

The code relies on gnuplot (\texttt{http://www.gnuplot.info}) to produce graphics.  Just install it
on your machine (apt, yum, rpm, dnf zypper,...) if you want to see them.  Otherwise you will
have to adapt the code to your favorite function plotter.  \texttt{gnuplot} is simple and old,
just send me your modifications for inclusion in a later release.




\section{Running the code}
\label{sec:run}

I assume you have a working directory separate from the source.  In the following I will call it
\texttt{atom-6.x.y/WORK}

This is a code with a long history.  Keeping it back-compatible (the original code still runs fine!)
makes it easier to check that modifications did not introduce bugs, at the cost of some compromises,
mainly in the number and format of files.

The old code and new additions communicated through files.  The new code still uses the same file formats
of several decades ago.
The good news about the present version, is that except fot their presence in the working directory
you do not have to worry about them in most cases.


\subsection{The all-electron structure}
\label{sec:run-ae}

The starting point of a pseudopotential construction is an all-electron calculation.
This is done by running the ``atom'' code:

\noindent\texttt{atom-6.x.y/WORK\$ ../atom.exe}

the program searches for a file \texttt{atom.dat} in the \texttt{WORK} directory with the information on the atomic
configuration and options of the calculation such as the exchange and correlation model.
This is the key input file and is described on a separate document adapted from the SIESTA
documentation.

If the file \texttt{atom.dat} is not present, the code asks the user the chemical symbol of the element to be calculated,
writes a \texttt{atom.dat} file with tdefault values and runs the calculation.
The user may edit the \texttt{atom.dat} file to change the default choices, taking into account that the
file is for fixed format read (back-compatibility).

In \texttt{atom-6.x.y/Doc/atom{\_}table.dat} the user can find an example of input file for elements H to Pu.

At the end of the calculation a file \texttt{datafile.dat} is written with the information of the 
atomic configuration and self-consistent potential.


\subsection{Constructing the pseudopotential}
\label{sec:run-psd}

The key operation is the construction of the pseudopotential.  To generate the Troullier-Martins
pseudopotential run

\noindent\texttt{atom-6.x.y/WORK\$ ../psd{\_}gen.exe}

The program reads the all-electron results from the \texttt{datafile.dat}, suggested core radii from the
\texttt{atom.dat} file, and presents that information plus data on maxima and last node of the
valence wave-functions to the user and ask if it accepts the defaults.  In the neagative case the user
chooses new core radii.

At the end of the calculation the user has three new files (back compatibility), an unformatted
\texttt{pseudo.dat} file used later in tests and compatible with SIESTA,
a \texttt{plot.dat} file used for later plotting, and
a formatted \texttt{psd.pot} file compatible with parsec and used for the Kleinman-Bylander transformation.
This last file is therefore the key output file of the program.


\subsection{Visualizing the pseudopotential}
\label{sec:run-show}

If \texttt{gnuplot} is installed the user can compare pseudo and all-electron wave-functions,
observe the pseudopotential and respective Fourier transforms by running

\noindent\texttt{atom-6.x.y/WORK\$ ../plot{\_}show.exe}

The program will read \texttt{plot.dat} (or another file chosen by the user) and allow
the user to change the range of the axis of the plots, and save a \texttt{pdf} version of
the plots.  If the user doesn't clean the \texttt{*.gp} files at the last question,
the \texttt{WORK} directory will have several \texttt{command{\_}}{\it name}\texttt{.gp} and
{\it name}\texttt{.gp} files.  The user can see the plots again with gnuplot

\noindent\texttt{atom-6.x.y/WORK\$ command{\_}}{\it name}\texttt{.gp}

and can edit the command file to change the plot.  The user can also use the data in {\it name}\texttt{.gp} files
to generate plots with other software, including xmgrace.


\subsection{Applying the Kleinman-Bylander transformation.}
\label{sec:run-KB}

In plane-wave codes the pseudopotential is used in its Kleinman-Bylander form.  That transformation
plus the generation of an efficient atomic basis set can be achieved by

\noindent\texttt{atom-6.x.y/WORK\$ ../kb{\_}conv.exe}

the program will read the \texttt{psd.pot} file and ask the users if he accepts the defaults and
in the negative case allows the choice of new parameters.

The key decision here is the choice of the local component of the pseudopotential.  A bad choice
can introduce ``ghost states'' and the code will check if they appear.

At the end of the calculation the program writes three new files (back compatibility), an unformatted
\texttt{pseudo{\_}kb.dat} used later in tests,
a \texttt{plot{\_}kb.dat} file used for later plotting, and
a formatted {\it Xy}\texttt{{\_}POTKB{\_}F.DAT} file compatible with the \texttt{cpw2000 code} where  {\it Xy}
is the chemical symbol of the relevant element.

The program also suggests the range of energy cutoffs for the plane-wave calculations.  It is a range because
different properties converge at different rates.


\subsection{Visualizing the KB operators.}
\label{sec:run-show-KB}

If \texttt{gnuplot} is installed the user can compare the pseudo wave-functions with the atomic basis functions,
observe the KB operators and respective Fourier transforms, and have an estimate of the rate 
of convergence of the total energy with plane-wave basis size by running

\noindent\texttt{atom-6.x.y/WORK\$ ../plot{\_}kb{\_}show.exe}

The program will read the \texttt{plot{\_}kb.dat} file or another file specified by the user.  See the section on
visualizing the pseudopotential for dealing with the files generated by the program.


\subsection{Logarithmic derivatives plots.}
\label{sec:run-show-ln}

If \texttt{gnuplot} is installed the user can generate the plots of the logarithmic derivatives of the
wave-functions with respect to energy for a given radius with

\noindent\texttt{atom-6.x.y/WORK\$ ../plot{\_}ln{\_}show.exe}

The program will read the \texttt{pseudo.dat}, \texttt{pseudo{\_}kb.dat} and \texttt{datafile.dat} files (or the files specified
by the user).   See the section on
visualizing the pseudopotential for dealing with the files generated by the program.


\subsection{Testing the pseudopotentials}
\label{sec:run-test}

By editing the \texttt{atom.dat} file and running

\noindent\texttt{atom-6.x.y/WORK\$ ../atom.exe} 

the user can either see the all electron results for a
different configuration (\texttt{ae} keyword on the first line and deleting the second line, back-compatibility
as you should have guessed) or see the result of a different configuration on the pseudopotential
described in \texttt{pseudo.dat}.  Running

\noindent\texttt{atom-6.x.y/WORK\$ ../atom{\_}kb.exe} 

the user achieves the same type of test for the KB pseudopotential described by \texttt{pseudo{\_}kb.dat}.

A series of tests of ionization and promotion energies can be done with the original \texttt{atom.dat}
by running

\noindent\texttt{atom-6.x.y/WORK\$ ../atom{\_}test.exe} 

At the end the program will print a comparison of the ionization energies and promotion energies
with all- electrons pseudopotential and pseudopotential in KB form.

Discrepancies on the range of a few hundreth of eV are to be expected, anything larger
may indicate a bad paseudopotential.


\subsection{Running all of it at once}
\label{sec:run-all}

Assuming you have gnuplot installed and you copied the executables to the code root, you can run 
all the code at once with a simple command

\noindent\texttt{atom-6.x.y/WORK\$ ../atom{\_}all.exe Si}

where you can swap the chemical symbol of my prefered element for the chemical symbol of any element
that interests you from Hydrogen to Oganesson (https://doi.org/10.1103/PhysRevB.26.4199).

You will get several pdf files, and many other files.  Those are described in the previous subsections.

\end{document}
