% Description of the atom.dat file
%
% To generate the printed version:
%
% pdflatex manual
% pdflatex manual
%
%
\documentclass[11pt]{article}
%\usepackage{makeidx}
%\usepackage{graphicx}

\tolerance 10000
\textheight 24cm
\textwidth 16cm
\oddsidemargin 1mm
\topmargin -20mm

%\makeindex

\parindent=0cm
\baselineskip=12pt
\parskip 5pt

\newcommand{\gpfig}[2][1.0]{
\includegraphics[angle=-90,width=#1\textwidth] {#2}
}

\newcommand{\singlefig}[2][1.0]{
\begin{figure}
\centering
\gpfig[#1]{#2}
\end{figure}
}

\begin{document}

% TITLE PAGE
% --------------------------------------------------------------

\begin{titlepage}

\begin{center}
~
\vfill
\vspace{1cm}
{\Huge {\bf The atom.dat file}}
\par\vspace{3cm}
\hrulefill
\par\vspace{3cm}
{\Large {\bf Version 3.2.2, January 2006}}
\par\vspace{2cm}
\hrulefill

{\Large Alberto Garc\'{\i}a \\
 Universidad del Pa\'{\i}s Vasco, Bilbao, SPAIN 

  wdpgaara@lg.ehu.es\\
  
Adapted by Jos\'e Lu\'{\i}s Martins, October 2021.
  }
\vfill
\end{center}

\end{titlepage}
% END TITLE PAGE
% --------------------------------------------------------------

\tableofcontents
\newpage

\section{PREFACE}

{\sc ATOM} is the name of a program originally written (circa 1982) by
Sverre Froyen at the University of California at Berkeley, modified
starting in 1990 by Norman Troullier and Jose Luis Martins at the
University of Minnesota, and currently maintained by Alberto Garcia
(wdpgaara@lg.ehu.es), who added some features and made substantial
structural changes to the April 1990 (5.0) Minnesota version while at
Berkeley and elsewhere.

Jose Luis Martins is maintaining his own version of the code:
\begin{verbatim}{\tt http://bohr.inesc.pt/~jlm/pseudo.html\end{verbatim}


The program's basic capabilities are:

\begin{itemize}
\item All-electron DFT atomic calculations for arbitrary electronic
configurations.

\item Generation of ab-initio pseudopotentials (several flavors).

\item Atomic calculations in which the effect of the core is represented
by a previously generated pseudopotential. These are useful to make
sure that the pseudopotential correctly reproduces the all-electron
results for the valence complex.

\end{itemize}



\section{A PRIMER ON AB-INITIO PSEUDOPOTENTIALS}

Time constraints prevent the inclusion of this section in this first
release of the {\sc ATOM} manual. But, in this case more than ever,
there is a lot to be gained from reading the original literature...
Here are some basic references:

\begin{itemize}
\item Original idea of the ab-initio pseudopotential:

Kerker, J. Phys. C 13, L189-94 (1980)\\
Hamann, Schluter, Chiang, Phys. Rev. Lett. 43, 1494 (1979)

\item More on HSC scheme:

Bachelet, Schluter, Phys. Rev. B 25, 2103 (1982)\\
Bachelet, Hamann, Schluter, Phys. Rev. B 26, 4199 (1982)

\item Troullier-Martins elaboration of Kerker method:

Troullier, Martins, Phys. Rev. B 43, 1993 (1991)\\

\item Core corrections:

Louie, Froyen, Cohen, Phys. Rev. B 26, 1738 (1982)

\item The full picture of plane-wave pseudopotential ab-initio calculations:

W. E. Pickett, ``Pseudopotential Methods in Condensed Matter
Applications'', Computer Physics Reports 9, 115 (1989)

M. C. Payne, M. P. Teter, D. C. Allan, T. A. Arias and
J. D. Joannopoulos, ``Iterative minimization techniques for ab initio
total-energy calculations: molecular dynamics and conjugate
gradients'', Rev. Mod. Phys. 64, 1045, (1992)

Also, the book by Richard Martin
``Electronic Structure: Basic Theory and Practical Methods''
(Cambridge University Press) has a chapter on pseudopotentials.

\item Use in {\sc SIESTA}:

J.M. Soler, E. Artacho, J.D. Gale, A. Garcia, J. Junquera, P. Ordejon,
D. Sanchez-Portal, ``The SIESTA method for ab initio O(N)
materials simulation'', Jour. Phys.: Condens. Matter, 14, 2745-2779
(2002). 

\end{itemize}

\section{COMPILING THE PROGRAM}

Edit {\tt make.inc} in the main source directory {\tt Src}, and modify, if
needed.

Type {\tt make}. After a short while you will have several executables
in that same directory. The program should work for any atom
without recompilation. 


\section{USING THE ATOM PROGRAM}


\subsection{All-electron calculations}

Assume we want to find the orbital eigenvalues, total energy, and/or
charge density of Si in its ground state. You should now go to a ``working'' 
directory and try the following:  Our
input file is named {\tt atom.dat} and contains the lines (see 
Sect.~\ref{sec:inputfile} for more details):

\begin{verbatim}
   ae     Si   ground state all-electron
 n=Si c=ca 
      14.0       0.0       0.0     120.0       6.0      80.0
    3    2
    3    0     2.00      0.00
    3    1     2.00      0.00

#2345678901234567890123456789012345678901234567890      Ruler
\end{verbatim}

We can run the calculation by typing
\textit{pathtoexecutables}\texttt{/atom.exe $>$ atom.out},
where we are redirecting the output to \texttt{atom.out}.


It is interesting to peruse the \texttt{atom.out} file.
In particular, it lists the orbital eigenvalues (in Rydbergs, as almost
every other energy in the program):

\begin{verbatim}
 nl    s      occ         eigenvalue    kinetic energy      pot energy

 1s   0.0    2.0000    -130.36911227     183.01377610    -378.73491457
 2s   0.0    2.0000     -10.14892691      25.89954256     -71.62102165
 2p   0.0    6.0000      -7.02876265      24.42537869     -68.74331196
 3s   0.0    2.0000      -0.79662738       3.23745216     -17.68692614
 3p   0.0    2.0000      -0.30705175       2.06135786     -13.62572529
\end{verbatim}

(For a relativistic or spin-polarized calculation, there would be
``up'' and ``down'' flags in the {\tt s} column above.)

\subsection{Pseudopotential generation}

To generate a pseudopotential for Si, using
the Troullier-Martins scheme repeat the calculation with relativistic 
effects (although they are quite small) and
use the LDA (Ceperley-Alder flavor). The input file
contains the lines (see Sect.~\ref{sec:inputfile} for more
details):

\begin{verbatim}
   pg     Si     Silicon
        tm2
 n=Si c=car 
      14.0       0.0       0.0     120.0       6.0      80.0
    3    4
    3    0     2.00      0.00
    3    1     2.00      0.00
    3    2     0.00      0.00
    4    3     0.00      0.00
    1.90     1.90     1.90     1.90
   pg      Silicon
                                     
#23456789012345678901234567890123456789012345678901234567890      Ruler
\end{verbatim}

Note the two extra lines with respect to an all-electron calculation.
The pseudopotential core radii for all channels are 1.90 bohrs. Even
though they are nominally empty in the ground state, we include the
$3d$ and $4f$ states in order to generate the corresponding
pseudopotentials. 

We can run the calculation by typing again 
\textit{pathtoexecutables}\texttt{/atom.exe $>$ atom.out}

The {\tt atom.out} file now has the spin-orbit effect and we
see in the eigenvalues that is small.  The empty scattering orbitals
may have a small positive eigenvalue.

\begin{verbatim}
 nl    s      occ         eigenvalue    kinetic energy      pot energy

 1s   0.5    2.0000    -130.51541377       0.00000000    -380.32924747
 2s   0.5    2.0000     -10.18537055       0.00000000     -71.99156489
 2p  -0.5    2.0000      -7.05518951       0.00000000     -69.08320077
 2p   0.5    4.0000      -7.00775181       0.00000000     -68.71538886
 3s   0.5    2.0000      -0.79937163       0.00000000     -17.74263422
 3p  -0.5    0.6667      -0.30807126       0.00000000     -13.66178946
 3p   0.5    1.3333      -0.30567131       0.00000000     -13.60785806
 3d  -0.5    0.0000       0.00231324       0.00000000      -0.38199583
 3d   0.5    0.0000       0.00231324       0.00000000      -0.38199582
 4f  -0.5    0.0000       0.00340155       0.00000000      -0.35430233
 4f   0.5    0.0000       0.00340155       0.00000000      -0.35430233
\end{verbatim}

Running the pseudopotential generation by typing
\textit{pathtoexecutables}\texttt{/psd{\_}gen.exe $>$ psd.out}
will give you (just accept the defaults) a new default output
where you can find the pseudopotential eigenvalues.

\begin{verbatim}
 nl    s      occ         eigenvalue    kinetic energy      pot energy

 1s   0.5    2.0000      -0.79936317       0.50555149      -3.74109919
 2p  -0.5    0.6667      -0.30805322       0.77243757      -3.26354801
 2p   0.5    1.3333      -0.30566087       0.76702417      -3.25195654
 3d  -0.5    0.0000       0.00231324       0.00233363      -0.10914167
 3d   0.5    0.0000       0.00231324       0.00233363      -0.10914167
 4f  -0.5    0.0000       0.00340158       0.00342993      -0.10122710
 4f   0.5    0.0000       0.00340158       0.00342993      -0.10122710
\end{verbatim}
which are almost identical to the all-electron values.


\subsubsection{Core Corrections}
\label{sec:cc}
The program can generate pseudopotentials with the non-linear
exchange-correlation correction proposed in S. G. Louie, S. Froyen,
and M. L. Cohen, Phys. Rev. B 26, 1738 (1982).

In the traditional approach (which is the default for LDA
calculations), the pseudocore charge density equals the charge density
outside a given radius $r_{pc}$, and has the smooth form
$$
\rho_{pc}(r) = A r   \sin(b r)
$$
inside that radius. A smooth matching is provided with suitable $A$ 
and $b$ parameters calculated by the program.

A new scheme has been implemented to fix some problems in the generation
of GGA pseudopotentials. The smooth function is now
$$
\rho_{pc}(r) =  r^2  \exp{(a + b r^2 +c r^4)}
$$
and derivatives up to the second are continuous  at $r_{pc}$.

To use core corrections in the pseudopotential generation
the jobcode in the first line should be {\tt pe} instead of {\tt pg}.

The radius $r_{pc}$ should be  given in the sixth slot in the last
input line (see above). If it is negative or zero (or blank), the
radius is then computed using the fifth number in that line ({\tt
rcore\_flag}, see the example input file above)
and the following criterion: at $r_{pc}$ the core charge density 
equals {\tt rcore\_flag}*(valence charge density).
It is {\it highly recommended} to set an explicit value for the pseudocore
radius $r_{pc}$, rather than letting the program provide a default.

If {\tt rcore\_flag} is input as negative, the full core charge is used.
If {\tt rcore\_flag} is input as zero, it is set equal to one, which will be
thus the default if {\tt pe} is given but no numbers are given for these
two variables.

The output file contains the radius used and the $A$ ($a$) and $b$ (and $c$)
parameters used for the matching.




\section{APPENDIX: THE INPUT FILE}
\label{sec:inputfile}

For historical reasons, the input file is in a rigid column
format. Fortunately, most of the column fields line up, so the
possibility of errors is reduced.  We will begin by describing in
detail a very simple input file for an {\bf all-electron calculation} for
the ground state of Si. More examples can be found in the {\tt
Tutorial} directory.

The file itself is:
\begin{verbatim}
   ae     Si   ground state all-electron
 n=Si c=ca 
       0.0       0.0       0.0       0.0       0.0       0.0
    3    2
    3    0     2.00      0.00
    3    1     2.00      0.00

#2345678901234567890123456789012345678901234567890      Ruler
\end{verbatim}

\begin{itemize}
\item The first line specifies:
	\begin{itemize}
	\item The calculation code ({\tt ae} here stands for ``all-electron'').
	\item A title for the job (here {\tt Si ground state all-electron}).
	\end{itemize} 
	(format 3x,a2,a50)

\item Second line:
	\begin{itemize}

	\item Chemical symbol of the nucleus (here {\tt Si}, obviously)
	\item Exchange-correlation type. Here, {\tt ca} stands for
          Ceperley-Alder.  The
          ``best'' LDA choice should be {\tt ca}.  It is also possible
          to use a gradient-corrected functional: {\tt pb} indicates use
          of the PBE scheme by Perdew, Burke, and Ernzerhof (PRL 77,
          3865 (1996)).
	  The exchange-correlation energy and potential
	  are computed using the Soler-Balb\'as package. 
   
	\item The character {\tt r} next to {\tt ca} is a flag to perform the
          calculation relativistically, that is, solving the Dirac equation
          instead of the Schrodinger equation. 
	  The full range of options is:
	    \begin{itemize}
		\item {\tt s} : Spin-polarized calculation, non-relativistic.
		\item {\tt r}: Relativistic calculation, obviously polarized.
		\item (blank) : Non-polarized (spin ignored), non-relativistic
           		calculation.
	    \end{itemize}
	\end{itemize}
	
	(format 3x,a2,3x,a2,a1,2x)

\item Third line. Its use is somewhat esoteric and for most
calculations it may contain just a 0.0 in the
position shown as the code will use the default values.

The first number is the nuclear charge, $Z$.  You can invent elements with fractional charge.
If it is zero the program will get it from the chemical symbol of the previous line.
The second and third number are there for historical reasons.  Their values are
never used.  The fourth to sixth numbers specify the integration mesh.
Again if they are zero the code uses default values.
The fourth number is the maximum range of the radius values $r_{\rm max}$.  
The sixth number $B$ controls the increment between consecutive grid points.
$r(n+1) \sim (1+1/B) r(n)$, in other words it takes about $B$ points to increase the
radius in the mesh by about $e$.  The fifth number $A$ controls the first non-zero
point in the mesh. $r(1) = 0$, $r(2) \simeq exp(-A)/(Z*B)$.

The full specification of the quasi-logarithmic mesh is
$a = exp(-A)/Z$, $b = 1/B$, $r(i) = a*(exp(b*(i-1))-1)$, $r(i) < r_{\rm max}$.

\end{itemize}

The rest of the file is devoted to the specification of the electronic
configuration:

\begin{itemize} 

\item Fourth line:\\
	 Number of core and valence orbitals. For example, for Si, we
	have $1s$, $2s$, and $2p$ in the core (a total of 3 orbitals), and
	$3s$ and $3p$ in the valence complex (2 orbitals).

	(format 2i5)

\item Fifth, sixth... lines: (there is one line for each valence
orbital)
	\begin{itemize}
	\item {\tt n} (principal quantum number)
	\item {\tt l} (angular momentum quantum number)
	\item Occupation of the orbital in electrons. 
	\end{itemize}

	(format 2i5,3f10.3)

	(There are two f input descriptors to allow the input of ``up''
	and ``down'' occupations in spin-polarized calculations (see
	example below)).
	There is a third float point number that can fix the eigenvalue of the orbital.
	It has not been used in years,  Forget it exists.

\end{itemize}

	
The different treatment of core and valence orbitals in the input for an
all-electron calculation is purely cosmetic. The program ``knows'' how
to fill the internal orbitals in the right order, so it is only
necessary to give their number. That is handy for heavy atoms...
Overzealous users might want to check the output to make sure that the
core orbitals are indeed correctly treated.

For compatibility with the old code, the file should end
with an empty line.  The new code doesn't care about it.

For a {\bf pseudopotential test calculation}, the format is exactly
the same, except that the job code is {\tt pt} instead of {\tt ae}. 


For a {\bf pseudopotential generation run}, in addition to the
electronic configuration chosen for the generation of the
pseudopotentials (which is input in the same manner as above), one has
to specify the ``flavor'' (generation scheme) and the set of core
radii $r_c$ for the construction of the pseudowavefunction. Here is an
example for Si using the Troullier-Martins scheme:

\begin{verbatim}
# 
   pg Si Pseudopotencial
        tm2     2.00
   Si   ca
         0
    3    3
    3    0      2.00
    3    1      0.50
    3    2      0.50
      1.12      1.35      1.17       0.0       0.0       0.0
#
#23456789012345678901234567890123456789012345678901234567890   Ruler
---------------------------------------
\end{verbatim}

Apart from the {\tt pg} (pseudopotential generation) job code in the
first line, there are two extra lines:

\begin{itemize}
\item Second line: 

Flavor and radius at which to compute logarithmic
derivatives for test purposes. 

The flavor can be one of :
\begin{tabular}{ll}
	tm2	&Improved Troullier-Martins\\
\end{tabular}

The {\tt ker} and {\tt tm2} schemes can get away with larger $r_c$,
due to their wavefunction matching conditions.

(format 8x, a3, f9.3)

\item The last line (before the blank line) specifies:

\begin{itemize}
\item The values of the $r_c$ in atomic units (bohrs) for the $s$,
$p$, $d$, and $f$ orbitals (it is a good practice to input the valence
orbitals in the order of increasing angular momentum, so that there is
no possible confusion).

(format 4f10.5)  

\item Two extra fields (2f10.5) which are relevant only if non-local
core corrections are used (see Sect~\ref{sec:cc}).
\end{itemize}
\end{itemize}

In the {\tt hsc} example above, only $s$ ,$p$, and $d$ $r_c$'s are
given. Here is an example for Silicon in which we are only interested
in the $s$ and $p$ channels for our pseudopotential, and use the Kerker
scheme:

\begin{verbatim}
# 
   pg Si Kerker generation
        ker     2.00
   Si   ca
         0
    3    2
    3    0      2.00
    3    1      2.00
      1.80      1.80      0.00       0.0       0.0       0.0

#23456789012345678901234567890123456789012345678901234567890   Ruler
\end{verbatim}


This completes the discussion of the more common features of the input
file. See the Appendix~\ref{sec:directives} for more advanced options.

%\addcontentsline{toc}{section}{Index}
%\printindex

\end{document}






